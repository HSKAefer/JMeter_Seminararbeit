\documentclass[a4paper,12pt]{article}
\usepackage[ngerman]{babel}
\usepackage[T1]{fontenc}
\usepackage[utf8]{inputenc}
\usepackage{graphicx}
\usepackage{lmodern}
\usepackage{float}
\usepackage[hyphens]{url} %hyphen for too long urls in the bibiography
\usepackage{listings}
\usepackage[nottoc]{tocbibind} %includes the bibtex bibliography into toc
\usepackage{xcolor} %color instead of color for the definecolor >html<
\usepackage{kpfonts}
\usepackage{natbib}
\usepackage[colorlinks=false]{hyperref}
\usepackage[hang]{footmisc}
\usepackage{fancyhdr}	%package for calling \pagestyle{fancy} 
\usepackage{ifthen} %to set the chaptertitle not in the tocs
\usepackage{titletoc}

\makeatletter
\author{Daniel Sch\"afer} \let\Author\@author
\makeatother

\pagestyle{fancy}
\fancyhf{}
%\fancyhead[RE]{\scshape\nouppercase{\leftmark}}	%LE - left side of the header for even pages, and right side for odd pages
\fancyhead[LO]{\scshape\nouppercase{\rightmark}}
\fancyhead[RO]{\thepage}

\setlength{\textheight}{598pt} %origin 598 in a4
\setlength{\headsep}{20pt}
\setlength{\headheight}{16pt}
\setlength{\topmargin}{0.46cm} %origin a4 = 17pt
\setlength{\marginparwidth}{38pt}
\setlength{\parindent}{0mm}	%neuer paragraph wird nicht eingerückt
\setlength{\parskip}{1mm}	%abstand nach unten hin
\setlength{\skip\footins}{1cm} %before footnote space
\setlength{\footnotemargin}{1em} %standardvalue = 1.8em
\setlength{\intextsep}{20pt plus2pt minus 4pt}	%space between text and the float environment like a picture. 
\setlength{\floatsep}{20pt plus2pt minus 4pt} %two floats behind space between them

\definecolor{graybackgroundColor}{HTML}{E5E5E5} %gray!20 for gray!30 use D9D9D9 for gray!10 F2F2F2

\hypersetup{ %required package hyperrref
  colorlinks   = true, %Colours links instead of ugly boxes
  urlcolor     = blue, %Colour for external hyperlinks
  linkcolor    = black, %Colour of internal links
  citecolor   = black %Colour of citations
}

\lstset{
aboveskip=20pt plus2pt minus 4pt,
belowskip=20pt plus2pt minus 4pt,
	numbers=left,
	numbersep=5pt,
	sensitive=false, %groß kleinschriebung
	tabsize=2,
	backgroundcolor=\color{graybackgroundColor},
	showspaces=false,
  showstringspaces=false,
  showtabs=false,
	breaklines=true, %falls der code über den rand herausragt!
		basicstyle=\footnotesize\ttfamily,
	numberstyle=\tiny\ttfamily,
	captionpos=b,
}

\lstdefinestyle{Cpp} {
language=C++
}
%---------------------------------------------------------------------------------------

\begin{document}

\begin{titlepage}
	\centering
	\includegraphics[width=0.7\textwidth]{bilder/hskalogo}\par\vspace{1cm}
	\vspace{0.5cm}
	{\Large Studiengang Informatik (Master)}\par
	\vspace{2cm}
	{\huge\bfseries Last-Test von Web-Seiten mit JMeter}\par
	\vspace{2cm}
	{\Large\bfseries Seminararbeit - Ausarbeitung}\par
	\vspace{0.5cm}
	\vspace{3cm}
	
	{\Large Daniel Schäfer (60118)\par}
	\vspace{1cm}
	{\Large\textbf{Betreuer:} Prof.~Dr.~Holger Vogelsang} 
	\vfill
	{\Large Sommersemester 2018\par}
	\vfill
% Bottom of the page
%{\large \today\par}
\end{titlepage}

\pagenumbering{roman}
\setcounter{page}{2}
\thispagestyle{empty}
\tableofcontents
\pagebreak
%\addcontentsline{toc}{section}{Abbildungsverzeichnis}
%\listoffigures

\pagenumbering{arabic}
\setcounter{page}{4}

\section{Einführung}
Mit dieser Ausarbeitung zur Seminararbeit wird das Performance und Last-Test Programm Apache JMeter ausführlich unter die Lupe genommen. Es werden seine Funktionen und Möglichkeiten erläutert und kurz Alternativen dazu erwähnt. Zusammenfassend werden Vor- und Nachteile aufgelistet, sowie ein Fazit gegeben.

\subsection{Motivation}
Man erlebt es immer wieder, dass Webseiten unter zu hoher Last in die Knie gezwungen werden. Sei es nun Amazon, Netflix und Konsorten durch DDoS-Angriffe \cite{online:AmazonDDoS}, oder auch hochschulinterne Seiten, die durch gegebene Anmeldefristen eine hohe Anzahl von gleichzeitigen Benutzern zu bewältigen haben. 

Gerade letzteres Szenario lässt sich im voraus gut vermeiden, da man die grobe Anzahl der Studierenden kennt und somit präventiv die Webseite auf die eingehende Last testen kann. Dabei werden sehr viele nebenläufige Anfragen an einen Anwendung gestartet und die Response Zeiten ausgewertet. Diese Art von Tests nennt man Last oder Performance-Tests und wird in die Klasse der Systemtests kategorisiert \cite{online:LastUndPerformanceTest}. 

Das frei verfügbare, unter Public License Key stehende Java Programm JMeter bietet diese und weitere Funktionalität um die Leistung einer Webseite bis ins kleinste Detail zu analysieren, testen, auswerten und visualisieren.

\subsection{Manuelle vs automatisierte Tests}
Bevor es an das eigentliche Thema JMeter geht, gibt es einen kleinen Exkurs in die sogenannten Systemtests. Diese wurden im vorherigen Abschnitt kurz erwähnt und haben die Besonderheit, dass sie sowohl manuell, als auch automatisiert ausgeführt werden können. 

\subsubsection{Manuelles Testen}
Beim manuellen Testen werden bestimmte Aktionen und Request von mehreren Benutzern nach bestimmten Vorgaben gestartet und die resultierenden Ergebnisse protokolliert. Es liegt auf der Hand, dass diese Art des testens ziemlich zeit -und resourcenaufwändig ist. Zusätzlich muss das Personal, sprich die Tester organisiert, betreut und verpflegt werden. Falls es sich dabei um die Programmierer selbst handelt, stehen diese im Zeitraum auch nicht für andere Tätigkeiten zur Verfügung. Der Faktor Mensch spielt natürlich bei dieser Art der Tests eine Rolle, wodurch nicht ausgeschlossen werden kann, dass Fehler während der Testphasen gemacht und diese entsprechend nicht erfasst werden. \cite[p. 11]{book:ApacheJMeter}

Trotz dieser Nachteile sollte man auf zusätzliche manuelle Tests\footnote{Das sogenannte User Acceptance Testing (UAT) ist gerade im finalen Entwicklungsstadium einer Anwendung unabdingbar. \cite{online:wikiUAT}} einer Anwendung nie verzichten, da eine "`reale"' Person über den Tellerrand schauen kann und es dadurch möglich ist, diverse Bugs ausfindig zu machen, die mit dem eigentlichen Test unter umständen gar nichts zu tun haben.

\subsubsection{Automatisiertes Testen}
Automatisiertes Testen funktionieren immer dann recht gut, wenn häufige Wiederholungen auftreten. Beispielsweise bei Regressiontests. Aber auch bei Lasttests ist die Testautomatisierung ein gängiges Verfahren. Die Tests laufen in erster Linie mit Software, was zur Folge hat, dass sie ziemlich statisch sind und dadurch beispielsweise keine Ästhetik geprüft werden kann, wie dies etwa bei menschlichen Testpersonen der Fall wäre.

Jedoch haben automatisierte Tests den großen Vorteil, dass sie kosteneffizienter sind. Man benötigt lediglich ein Budget für etwaige Lizenzgebühren der Software oder für den Support. Des weiteren ist man durch Testsoftware in der Lage, sehr viele Benutzer gleichzeitig zu erzeugen und parallel auf ein System zugreifen zu lassen. Mit Testskripten sogar rund um die Uhr \cite{online:Testautomatisierung}. Dadurch werden Systeme bis zur ihren Grenzen und darüber hinaus getestet.

\section{Apache JMeter}
Auf den ersten Blick erscheint Apache JMeter wie das Relikt aus einer längst vergangenen Zeit. Man sollte sich jedoch von der veralteten und trägen Swing-basierten GUI nicht täuschen lassen, denn hinter der Oberfläche steckt ein mächtiges Werkzeug, mit dem sich alle Arten von Tests erstellen und ausführen lassen. \cite{online:heiseJMeterOderGatling}

Tatsächlich stammt JMeter aus einer Zeit als das Web und Application Server noch in den Kinderschuhen steckten. Ursprünglich wurde JMeter entwickelt um den Application Server Tomcat zu testen. Seitdem wurde das Programm fortlaufend weiter entwickelt, so dass man heute in der Lage ist von verteilte Tests in der Cloud, über das Testen von dynamischen Webseiten bis hin zu JUnit Tests, alle möglichen Arten von Tests zu realisieren. \cite{online:ApacheJMeter}

 
\includegraphics[width=1\textwidth]{bilder/jmeter_1.png}\par\vspace{1cm}
JMeter Platformunabhängig wegen java anwendung. kann sowohl auf mac als auch auf windows laufen. lediglich die java runtime

Wann wurde das Tool das erste mal entwickelt? wie oft wird es eingesetzt. auch von großen unternehmen? kommt bei sap auch zum einsatz

\subsection{Was kann JMeter}
Es lassen sich ebenfalls mehrere Benutzer erzeugen und eine Anwendung mit diesen simultan testen.  
 
\subsubsection{Lastentests}

\subsubsection{Captcha Analyse}
Captcha analysieren und erkenne?

\subsubsection{Auto vervollständigung von Formularen}
befüllen von formularen

\subsubsection{JUnit Tests}
junit tests in jmeter importieren
\subsection{Installation}
\subsubsection{Setting up the environment}
\subsubsection{Running JMeter}

\subsection{JMeter GUI}
\includegraphics[width=1\textwidth]{bilder/jmeter_2.png}\par\vspace{1cm}

\subsection{JMeter Command Line}
GUI verbraucht viel Speicher. Nicht empfohlen für große Lastentests
Command Line ist geeignet für die Integration in andere Systeme wie Jenkins/Continious Integration
\subsubsection{Ausführen des Console Clients}
Zuerst sollte ein Testscript verfügbar sein. Man kann hier ein vorhandenes File verwenden oder in JMeter selbst ein neues *jmx-File erzeugen. Minimale Voraussetzungen sind eine Thread Group sowie ein Sampler wie etwa ein HTTP Request mit entsprechender URI.

JMeter GUI beenden und Command Line starten. Danch in das /bin Verzeichnis von JMeter navigieren. Dort startet man den Test via dem Befehl:
\begin{lstlisting}[style=Cpp]
  jmeter -n -t [jmx file] -l [results file]
\end{lstlisting}
In der folgenden Tabelle sieht man einige häufig verwendete Parameter und ihre Bedeutung. Um eine Übersicht aller Parameter zu erhalten, kann man in der Console den jmeter -? ausführen.
\begin{table}[H]
	\centering
	\begin{tabular}{|l|l|}
		\hline
		\textbf{Parametername} & \textbf{Bedeutung} \\
		\hline
		-n & non gui mode \\
		-t & location of jmeter script - combined with -n \\
		-l & location of result file \\
		-? & alle Parameter an die zur Auswahl stehen \\
		-h & Beispielbefehle, die man verwenden kann \\
		-L & Log level - Wann soll etwas geloggt werden \\
		-e & Um html Reports zu erzeugen\\
		-o & Location of the output folder - combined with -e or -g \\
		-g & html report wird aus einem csv file erzeugt \\ 
		\hline
	\end{tabular}
	\caption[tab_parameter_non_gui]{Befehle der JMeter Command Line}
	\label{tab_parameter_non_gui}
\end{table}





\section{Der Testplan}
Was ist überhaupt ein Testplan
\subsection{Elemente eines Testplans}
Welche Elemente kann ein Testplan enthalten

1 Testplan anlegen
\subsubsection{Thread-Groups}
Ähnlich wie JUnit. Hier möglichkeiten setup teardown neben den eigentlichen gruppen
Gleichzeitige Benutzer usw.
\subsubsection{Sampler}
Sind die eigentlichen Tests
\subsubsection{Listener}
Elemente die Informationen über die Performance Tests enthalten
Zum abfragen bzw. ergebnisse visualisieren

werden verwendet um ergebnisse und metriken der tests anzuzeigen

\subsubsection{http request}
\subsubsection{Assertions}
Mit jmeter lassen sich assertions testen. Dazu wird der response von einem request untersucht assert 200. assert 500 etc.

\section{Lasttests von Webseiten}

\section{Funktionale Tests}

\section{Wissenswertes und Besonderheiten}
Man kann die Ergebnisse zur Laufzeit in Logfiles schreiben.

\subsection{JMeter Plugin Manager}



\subsection{Aufzeichnen von Aktionen - UI web Test}
Build in Solution:
Bei non-test-Elements Testscript Recorder <-hiermit kann man ein script aufzeichnen um ganze Webseiten zu untersuchen

\includegraphics[width=1\textwidth]{bilder/badboy.png}\par\vspace{1cm}
Für Windows only Platformabhängig BadBoy:
Ansonsten drittanbieter verwenden wie etwa badboy.

Chrome Plugin:
Blazemeter
Mit Blazemeter ist man in der Lage direkt in Chrome und somit platformunabhängig Aufzeichnungen von Aktionen auf Webseiten aufzunehmen.
Kleiner Nachteil: Man muss sich bei Blazemeter registrieren um den Export in .jmx Dateien ausführen zu können.
\includegraphics[width=0.5\textwidth]{bilder/blazemeter.png}\par\vspace{1cm}

\subsection{Datenbankabfragen - Datenbank testplan?}
Auch datenbankabfragen lassen sich damit visualisieren

Falls interersse besteht, wie es funktioniert bitte in die Ausarbeitung schauen.
Thread Gruppe erstellen -> Configuration (JDBC Connection Configuration)
MySQL jdbc jar in lib folder von jmeter adden 
DAnn Sampler -> JDBC Request and put in values.. z.b. select * from table


\subsection{Virtuelle Compute Unit}
In die Cloud um an einer zentralen stelle viele rechner zu bedienen?
Eventuell viele Rechner simulierenö

\subsection{JMeter change Settings}
In /bin folder von JMeter kann man die Datei user.properties öffnen und Werte anpassen.

\section{Nachteile von JMeter}
Wenn ein Variablenfeld required ist. z.b. bei JDBC Connection Configuration. dann wird einem der fehler nicht direkt im feld angezeigt.

Ziemlich resourcenhungrig. Trotz 16GB Arbeitsspeicher hängt sich das Ding gerne mal auf.
\section{Fazit}
Alles in allem ist JMeter ein Super tool

\pagebreak
\thispagestyle{empty}
\bibliographystyle{plain}
\bibliography{Bibliography}

\end{document}