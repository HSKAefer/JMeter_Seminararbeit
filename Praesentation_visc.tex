\documentclass[xcolor=dvipsnames]{beamer}
%\usetheme{Pittsburgh}
\usepackage{pgfpages}
\usepackage{graphicx}
\usepackage{colortbl}
\usepackage{tikz}
\usepackage{pgfplots}
\usepackage{booktabs}
\usepackage[scientific-notation=true]{siunitx}
\usetikzlibrary{decorations.pathreplacing}
%\usetheme{Darmstadt}
%\usetheme{CambridgeUS}
\usetheme{MarburG}
%\usetheme{Dresden}
%\usetheme{Rochester}
%\usetheme{Madrid} 
%\usetheme{AnnArbor}
%\usetheme{Bergen}
%\usetheme{Antibes}
%\usetheme{JuanLesPins}
%\usetheme{Singapore}
\usecolortheme{whale}
%\usecolortheme{beaver, albatross, whale}
%\setbeameroption{show notes on second screen=left}
%\usetheme{Copenhagen}
%\usetheme{Berkeley}
%\usetheme{Goettingen}
%\usetheme{Hannover}
\usepackage[ngerman]{babel}
\usepackage[utf8]{inputenc}
\newcommand{\pro}{\item[\boldmath${\color{green}+}$ ]}
\newcommand{\con}{\item[\boldmath$ {\color{red}-}$ ]}
\usetikzlibrary{shapes.misc}
\usetikzlibrary{calc}
\tikzset{cross/.style={cross out, draw=red, minimum size=2*(#1-\pgflinewidth), inner sep=0pt, outer sep=0pt},
%default radius will be 1pt. 
cross/.default={1mm}}

\let\oldfootnotesize\footnotesize
\renewcommand*{\footnotesize}{\oldfootnotesize\tiny}

\newcommand{\tikzmark}[1]{\tikz[overlay,remember picture] \node (#1) {};}
\newcommand{\DrawBox}[1][]{%
    \tikz[overlay,remember picture]{
    \draw[red,#1]
      ($(left)+(-0.84cm,0.9cm)$) rectangle
      ($(right)+(-0.22cm,-0.2cm)$);}
}

%OPTIONAL - WENN TOC DABEISTEHT WEGLASSEN!!!!!
%\AtBeginSection[]{
%\frame{
%\frametitle{Agenda}
%\tableofcontents[currentsection]
%}
%}
%ENDE OPTIONAL

%Title definieren
\title{Performance Testing mit JMeter}
\subtitle{Seminararbeit SS 2018}
\titlegraphic {\includegraphics[width=4cm]{bilder/hskalogo}}
%\subtitle{Virtual Instruction Set Computer}
\author{Daniel Schäfer}
\date{\today}
%\institute{Seminararbeit SS 2016}

%Ende Titel

\begin{document}

\begin{frame}
\titlepage
\end{frame}

\begin{frame}
\frametitle{Agenda}
		\tableofcontents
\end{frame}

\section{Einleitung}
\begin{frame}\frametitle{Einleitung}
	Historischer Verlauf von Prozessoren + \emph{Death of CPU Scale}
%\begin{figure}
%\includegraphics[width=0.7\textwidth]{bilder/deathofcpu}
%\end{figure}
\begin{figure}
        \begin{tikzpicture}[scale=0.9]
				\draw [color=gray!50]  [step=5mm] (0,0) grid (7.6,5.1);
		\draw[color=blue, line width=2.5pt] plot[smooth]coordinates{(0.3,0.7)(1.5,1.2)(3,1.8)(4.5,2.7)(6,3.8)(7.5,5)};
			\draw[color=Green, line width=2.5pt] plot[smooth]coordinates{(0.3,0.4)(1.5,0.7)(3,1.2)(4.5,2)(6,2.2)(7.5,2.21)};	
				% Achsen zeichnen
		\draw[->,thick] (0,0) -- (7.7,0) node[right] {$\scriptstyle Jahr$};
		\draw[->,thick] (0,0) -- (0,5.2) node[above] {$\scriptstyle $};
		%\draw[rotate=90] (0,0) node[left] {$\scriptstyle Nanu$};
		%\node[rotate=90,left] at (-1,3) {$\scriptstyle Transistoren$};
		
		 \draw(-0.1,0) -- (0.1,0) node[left=4pt] {$\scriptstyle 0$};
		\draw(-0.1,1) -- (0.1,1) node[left=4pt] {$\scriptstyle 10^1$};
		\draw(-0.1,2) -- (0.1,2) node[left=4pt] {$\scriptstyle 10^3$};
		\draw(-0.1,3) -- (0.1,3) node[left=4pt] {$\scriptstyle 10^5$};
		\draw(-0.1,4) -- (0.1,4) node[left=4pt] {$\scriptstyle 10^7$};
		\draw(-0.1,5) -- (0.1,5) node[left=4pt] {$\scriptstyle 10^{10}$};
		
		
		\draw(0,-0.1) -- (0,0) node[below=4pt] {$\scriptstyle 1970$};
		\draw(1.5,-0.1) -- (1.5,0) node[below=4pt] {$\scriptstyle 1980$};
		\draw(3,-0.1) -- (3,0) node[below=4pt] {$\scriptstyle 1990$};
		\draw(4.5,-0.1) -- (4.5,0) node[below=4pt] {$\scriptstyle 2000$};
		\draw[color=red,style=dashed,line width=1pt](5.2,3.5) -- (5.2,-0.1) node[below=1pt] {$\scriptstyle 2004$};		
		\draw(6,-0.1) -- (6,0) node[below=4pt] {$\scriptstyle 2010$};
		\draw[color=red,style=dashed,line width=1pt](6.8,4.7) -- (6.8,-0.1) node[below=1pt] {$\scriptstyle 2016$};
		\draw(7.5,-0.1) -- (7.5,0) node[below=4pt] {$\scriptstyle 2020$};

	 %\draw (2.25,0) node[cross,rotate=15] {};
		%\draw (3.46,0) node[cross,rotate=15] {};
		%\draw (4.5,0) node[cross,rotate=15] {};
		%\draw (5.1,0) node[cross,rotate=15] {};
		
		\filldraw[color=red, nearly opaque] (2.25,1.8) -- (2,2) -- (1.2,2) -- (1.2,1.6) -- (2,1.6) --cycle;
		\filldraw[color=red, nearly opaque] (3.46,2.4) -- (3.21,2.6) -- (1.8,2.6) -- (1.8,2.2) -- (3.21,2.2) --cycle;
		\filldraw[color=red, nearly opaque] (4.5,3.0) -- (4.25,3.2) -- (3.6,3.2) -- (3.6,2.8) -- (4.25,2.8) --cycle;
		\filldraw[color=red, nearly opaque] (5.1,3.5) -- (4.85,3.7) -- (3.45,3.7) -- (3.45,3.3) -- (4.85,3.3) --cycle;
		\filldraw[color=red, nearly opaque] (6.6,4.6) -- (6.35,4.8) -- (4.5,4.8) -- (4.5,4.4) -- (6.35,4.4) --cycle;
		
		\draw (2.0,1.8)node[left]{$\scriptstyle 386$};
		\draw (3.3,2.4)node[left]{$\scriptstyle Pentium$};
		\draw (4.3,3.0)node[left]{$\scriptstyle P4$};
		\draw (5,3.5)node[left]{$\scriptstyle Itanium\;2$};
		\draw (6.5,4.6)node[left]{$\scriptstyle Octa\;Core\;i7$};
		
		
		\fill[yellow!10] (0.2,4.0) rectangle (3.5,5);
		\draw [color=blue, line width=1pt](0.4,4.75)--(0.6,4.75) node[right]{$\scriptstyle Transistoren\;(000)$};
		\draw [color=Green, line width=1pt](0.4,4.25)--(0.6,4.25) node[right]{$\scriptstyle Takt\;(MHz)$};
		
		%\foreach \x in {1,2,3,4,5}
			
			%\draw (\x,-0.1) -- (\x,0) node[below=4pt] {$\scriptstyle \x$};
    %\foreach \y in {0,1,2,3}
			%\draw (-0.1,\y) -- (0.1,\y) node[left=4pt] {$\scriptstyle \y$};
				\end{tikzpicture}	
\end{figure}


\begin{description}
\item [2004 -] Einführung Mehrkernprozessoren 
\item [2016 -] Leistungssteigerung im Singlethread Betrieb
%\item [2020 -] 10 Mrd. Transistoren
\end{description}
\end{frame}



\subsection{Idee hinter VISC}
\begin{frame}
\frametitle{Idee hinter VISC}
\includegraphics[width=1.0\textwidth]{bilder/cpu_auslastung}
\end{frame}

\begin{frame}
\frametitle{Idee hinter VISC}
\begin{columns}
\begin{column}{0.5\textwidth}
\emph{Multithreading}
\begin{figure}
        \begin{tikzpicture}
				\node at (-2.8,0.7) {\includegraphics[scale=0.05]{bilder/amd_logo}};
				\node at (-1.5,1) {\includegraphics[scale=0.05]{bilder/intel_logo}};
					\node at (-0.3,0.8) {\includegraphics[scale=0.05]{bilder/arm_logo}};
				\draw[->,color=black,line width=1.5pt](-3.2,0)--(-1.8,-1.5) node[right] {};
				\draw[->,color=black,line width=1.5pt](0.2,0)--(-1.2,-1.5) node[right] {};
				\draw[->,color=black,line width=1.5pt](-2.0,0)--(-1.6,-1.5) node[right] {};
				\draw[->,color=black,line width=1.5pt](-1.0,0)--(-1.4,-1.5) node[right] {};
				
				%\fill[brown] (-1.9,-1.6) rectangle (-1.2,-3.0);
				
				\fill[top color=Blue!90,bottom color=Blue!2,middle color=Blue!30,shading=axis,opacity=0.25
				,line width=1.5pt] (-1.5,-1.8) circle (0.4 and 0.1);
				%\fill[top color=Blue!50!black,bottom color=Blue!10,middle color=Blue,shading=axis,opacity=0.25,line width=1.5pt,line width=1.5pt] (-1.5,-2.8) circle (0.4 and 0.1);
				\fill[left color=Blue!50!black,right color=Blue!50!black,middle color=Blue!50,shading=axis,opacity=0.25] (-1.1,-1.8) -- (-1.1,-2.8) arc (360:180:0.4cm and 0.1cm) -- (-1.9,-1.8) arc (180:360:0.4cm and 0.1cm);
				
				\draw[Blue] (-1.5,-3.5)node[above]{$\scriptstyle 1\;Kern$};
				\draw (-2.2,-1.2)node[left]{$\scriptstyle 4\;Threads$};
								
				\end{tikzpicture}
				\end{figure}

\end{column}
\vrule{}
\begin{column}{0.5\textwidth}
\emph{Inverse Hyperthreading}
\begin{figure}
        \begin{tikzpicture}
				
				\node at (1,0) {\includegraphics[scale=0.2]{bilder/visc_logo}};
				
				\draw[line width=1.5pt,color=black](1,-0.6) -- (1,-1.1) node[right] {};
				\draw (1.2,-0.8)node[right]{$\scriptstyle 1\;Thread$};
				\draw[->,line width=1.5pt,color=black](1,-1.1) -- (-1,-2) node[right] {};
				\draw[->,line width=1.5pt,color=black](1,-1.1) -- (0.5,-2) node[right] {};
				\draw[->,line width=1.5pt,color=black](1,-1.1) -- (1.6,-2) node[right] {};
				\draw[->,line width=1.5pt,color=black](1,-1.1) -- (3,-2) node[right] {};
				
				
				\fill[top color=Blue!90,bottom color=Blue!2,middle color=Blue!30,shading=axis,opacity=0.25
				,line width=1.5pt] (-1,-2.3) circle (0.4 and 0.1);
				\fill[left color=Blue!50!black,right color=Blue!50!black,middle color=Blue!50,shading=axis,opacity=0.25] (-0.6,-2.3) -- (-0.6,-3.3) arc (360:180:0.4cm and 0.1cm) -- (-1.4,-2.3) arc (180:360:0.4cm and 0.1cm);
				
				\fill[top color=Blue!90,bottom color=Blue!2,middle color=Blue!30,shading=axis,opacity=0.25
				,line width=1.5pt] (0.3,-2.3) circle (0.4 and 0.1);
				\fill[left color=Blue!50!black,right color=Blue!50!black,middle color=Blue!50,shading=axis,opacity=0.25] (0.7,-2.3) -- (0.7,-3.3) arc (360:180:0.4cm and 0.1cm) -- (-0.1,-2.3) arc (180:360:0.4cm and 0.1cm);
				
				
				\fill[top color=Blue!90,bottom color=Blue!2,middle color=Blue!30,shading=axis,opacity=0.25
				,line width=1.5pt] (1.6,-2.3) circle (0.4 and 0.1);
				\fill[left color=Blue!50!black,right color=Blue!50!black,middle color=Blue!50,shading=axis,opacity=0.25] (2,-2.3) -- (2,-3.3) arc (360:180:0.4cm and 0.1cm) -- (1.2,-2.3) arc (180:360:0.4cm and 0.1cm);
				
				\fill[top color=Blue!90,bottom color=Blue!2,middle color=Blue!30,shading=axis,opacity=0.25
				,line width=1.5pt] (2.9,-2.3) circle (0.4 and 0.1);
				\fill[left color=Blue!50!black,right color=Blue!50!black,middle color=Blue!50,shading=axis,opacity=0.25] (3.3,-2.3) -- (3.3,-3.3) arc (360:180:0.4cm and 0.1cm) -- (2.5,-2.3) arc (180:360:0.4cm and 0.1cm);
				
				\draw(1.1,-4) node[below,color=Blue] {$\scriptstyle 4\;Kerne$};
				
				\end{tikzpicture}
				\end{figure}

\end{column}
\end{columns}

%\begin{figure}
 %       \begin{tikzpicture}
	%\draw [densely dashed] (0.5,0) -- (0.5,10);
%\end{tikzpicture}
	%			\end{figure}
%\begin{itemize}
%	\item Hyperthreading bei AMD, Intel, ARM
%\end{itemize}
%\begin{figure}
%\includegraphics[width=0.7\textwidth]{bilder/multithreading}
%\end{figure}
%\begin{itemize}
%	\item \emph{Inverse Hyperthreading} bei VISC
%\end{itemize}
\end{frame}



\begin{frame}
	\frametitle{Idee hinter VISC}
	\begin{figure}%
	\begin{tikzpicture}
	\node at (-2,4.2)  {\includegraphics[width=0.2\textwidth]{bilder/cisc_cpu}};
		\node at (3,4.4) {\includegraphics[width=0.2\textwidth]{bilder/risc_cpu}};
	\draw(-2,3) node {$\scriptstyle \textbf{CISC}$};
		\draw (2,3.5) node {$\scriptstyle \textbf{RISC}$};
		\node at (-2,2) [rotate=12] {\fbox{\includegraphics[width=0.25\textwidth]{bilder/cisc_mul}}};
	\node at (2,2) [rotate=-14] {\fbox{\includegraphics[width=0.25\textwidth]{bilder/risc_mul}}};
		\draw[rounded corners, fill=Blue!50] (-4,0) rectangle (4,-1); 
		\node at (0,-0.5) {$\scriptstyle VISC\;-\;Instruction\;Set\;Architecture$};
		\draw[->,line width=1pt] (-1.5,1.5) -- (-0.4,0);
		\draw[->,line width=1pt] (1.5,0.9) -- (0.7,0);
		
	\end{tikzpicture}
	
	\end{figure}
\end{frame}


%Präsentation Blatt 1
\subsection{VISC-Begriffsdefinition}
\begin{frame}
\frametitle{VISC-Begriffsdefinition}
\begin{description}
\item [\textbf{VISC -}] \textbf{V}irtual \textbf{I}nstruction \textbf{S}et \textbf{C}omputer
\end{description}
\begin{itemize}
\item Verbesserte IPC-Werte durch \emph{Inverse Hyperthreading}
	\item kann x86 und ARM Architekturen verarbeiten
\end{itemize}
\begin{figure}\footnote{http://www.softmachines.com/wp-content/uploads/2014/12/Soft-Machines-VISCtm-Architecture-Tech-Briefing-vF.pdf}
\includegraphics[width=0.25\textwidth]{bilder/visc}
\end{figure}
\end{frame}
%Ende Blatt 1

%beginn blatt 2
	\subsection{Soft Machines}
\begin{frame}{Soft Machines}
%\frametitle{Soft Machines} <-alternative schreibweise des frametitle
\begin{figure}
\includegraphics[width=0.7\textwidth]{bilder/sm}
\end{figure}
	\begin{itemize}
		\item<1-> Halbleiterhersteller aus Santa Clara - Kalifornien (Silicon Valley)
		\item Gründer Mohammad Abdallah kam von Intel 
		\item ca. 250 Mitarbeiter
		\item 7 Jahre Entwicklungszeit
		\item 80 Patente im Bereich VISC
		\item Budget von ca. \$125 Mio.
		\item Investoren wie AMD, Samsung, GlobalFoundries
	\end{itemize}
\end{frame}
%ende blatt 2

%beginn blatt 3

%ende blatt 3

%beginn blatt 4
\section{VISC-Architektur}
\begin{frame}
\frametitle{VISC-Architektur}
	\begin{center}
		\begin{tikzpicture}
			\draw [fill=Gray!50] (-3,3) rectangle (3,4);
			\node at (0,3.5) {$\scriptstyle [SW]\;Sequential\;Code$};
			\draw [->,line width=1pt, color=Blue] (0,3) -- (0,1.2);
			\node at (1.5,2.5) [color=Blue] {$\scriptstyle [SW]\;Single\;Thread$};
			\draw [rounded corners, color=Gray!50!Black] (-4.4,-3.3) rectangle (4.4,2);
				\draw[->,color=Blue!40, line width=1pt] plot[smooth]coordinates{(0,0.2)(-1.5,-0.2)(-2,-1)};
				\draw[->,color=Blue!40, line width=1pt] plot[smooth]coordinates{(0,0.2)(1.5,-0.2)(2,-1)};
				
			\draw [fill=Blue!40] (-2.5,1.2) rectangle (2.5,0.2);
			
			\node at (0,0.7) {$\scriptstyle [HW]\;Global\;Front\;End$};
	
			\draw [top color=White, bottom color=Blue!30] (-4,-2.6) -- (-3,-1.6) -- (-0.2,-1.6) -- (-1.2,-2.6) --cycle;
			\draw [top color=White, bottom color=Blue!30] (-0.2,-2.6) -- (0.8,-1.6) -- (4,-1.6) -- (3,-2.6) --cycle;
			
			
			\draw [fill=Blue!40](-4,-2) -- (-3,-1) -- (0.5,-1) -- (-0.5,-2) -- (-4,-2);
			\draw [fill=Blue!40] (-0.5,-2) -- (0.5,-1) -- (4,-1) -- (3,-2) -- (-3,-2);
			\draw [<->,line width=1pt,dashed,color=White] (-1,-1.5) -- (1,-1.5);
			
			
			
			\node at (-2,-1.5) [color=White]{$\scriptstyle Virtual\;Core\;1$};
		\node at (2,-1.5) [color=White]{$\scriptstyle Virtual\;Core\;2$};
				\node at (-2.3,1.7) [color=Gray!50!Black] {$\scriptstyle [HW]\;Virtual\;Cores\;(Blackbox)$};
				
				\node at (3,-0.5) [color=Blue!50] {$\scriptstyle [HW]\;Threadlets$};
				
				\node at (-2.3,-2.3) {$\scriptstyle Core\;1$};
				\node at (1.8,-2.3) {$\scriptstyle Core\;2$};
				
			
		\end{tikzpicture}
	\end{center}
\end{frame}

\begin{frame}
\frametitle{VISC-Architektur}
	\begin{center}
	\begin{figure}\footnote{http://www.softmachines.com/wp-content/uploads/2015/10/Soft-Machines-VISC-Linley-PC-7Oct-2015-Final-Presentation.pdf}
\includegraphics[width=0.95\textwidth]{bilder/visc_architektur}
\end{figure}
	\end{center}
\end{frame}	



\subsection{Virtual Software Layer}
\begin{frame}
\frametitle{Virtual Software Layer}
	\begin{enumerate}
		\item Softwareseitige Konvertierung von eingehender ISA in VISC-ISA
		\item Conversion Layer mit Legacy Modus 
		\item x86 und ARM Unterstützung soll möglich sein
	\end{enumerate}
\end{frame}	

\subsection{Global Front End}
\begin{frame}
\frametitle{Global Front End}
\begin{figure}
\footnote{http://www.softmachines.com/wp-content/uploads/2014/10/SMI\_Dual\_SW\_Threads-04.png}
\includegraphics[width=0.45\textwidth]{bilder/front_end}
\end{figure}
\begin{itemize}
\item Herzstück von Soft Machines R\&D Abteilung
\item Einzigartiges Stück Hardware
\item Hier entsteht die größte Leistungssteigerung
\item Gewichtung der Anwendungen in Heavy/Light Apps
\item Zerlegung SW-Threads in HW-Threadlets
\item Virtual Cores einbinden / Zuweisung an Kerne 
\end{itemize}
	
\end{frame}



\subsection{Virtual Cores}
\begin{frame}
\frametitle{Virtual Cores (Single Thread)}
 %Global Front End - Mit den virtuellen Kernen und Threadlets
\begin{figure}
        \begin{tikzpicture}
				
				\node at (0,3.4) {$\scriptstyle Heavy\;App$};
				\draw [->,line width=5pt,color=Blue!40] (0,3.1) -- (0,2.4);
				
				\draw [top color=White!50, bottom color=Blue!50] (-4,-0.5) -- (-3,1.5) -- (-0,1.5) -- (-1,-0.5) -- (-4,-0.5);
				\draw [top color=White!50, bottom color=Blue!50] (0,-0.5) -- (1,1.5) -- (4,1.5) -- (3,-0.5) -- (0,-0.5);
			
				\draw [fill=Blue!40] (-4,0) -- (-3,2) -- (4,2) -- (3,0) -- (-4,0);
				
				\node at  (0,1.2) [fill=Blue!40] {$\scriptstyle Virtual\;Core\;1$};
				
				\node at  (-2.5,-0.8) {$\scriptstyle Physical\;Core\;1$};
				\node at  (1.5,-0.8) {$\scriptstyle Physical\;Core\;2$};
				
				\draw [->,line width=1pt,style=dashed,color=white] (-0.2,0.8) .. controls (-2,0.6) .. (-3.5,-0.2);
				\draw [->,line width=1pt,style=dashed,color=white] (0,0.8) .. controls (-1,0.6) .. (-1.5,-0.2);
				\draw [->,line width=1pt,style=dashed,color=white] (0.3,0.8) .. controls (1,0.6) .. (2,-0.2);
				
				\node at (2,0.6) [color=white] {$\scriptstyle Threadlets$};
				
				\draw [rounded corners, color=Blue] (-4.3,-1.2) rectangle (4.3,4.4);
				\node at (-3,4) [color=Blue] {$\scriptstyle Virtual\;Cores$};	
			\end{tikzpicture}
			\end{figure}
\end{frame}	

	\begin{frame}
		\frametitle{Virtual Cores (Dual Threads)}
\begin{figure}
        \begin{tikzpicture}
				\node at (-1,3.4) {$\scriptstyle Heavy\;App$};
				\draw [->,line width=5pt,color=Blue!40] (-1,3.1) -- (-1,2.4);
				
				\node at (3.2,3.4) {$\scriptstyle Light\;App$};
				\draw [->,line width=1pt,color=Green!60] (3.5,3.1) -- (3.5,2.4);
				
					\draw [top color=White!50, bottom color=Blue!50] (-4,-0.5) -- (-3,1.5) -- (-0,1.5) -- (-1,-0.5) -- (-4,-0.5);
				\draw [top color=White!50, bottom color=Blue!50] (0,-0.5) -- (1,1.5) -- (3.2,1.5) -- (2.2,-0.5) -- (0,-0.5);
				
				\draw [top color=White, bottom color=Green!50] (2.2,-0.5) -- (3.2,1.5) -- (4,1.5) -- (3,-0.5) -- (2.2,-0.5);
				
			
				\draw [fill=Blue!40] (-4,0) -- (-3,2) -- (2.8,2) -- (1.8,0) -- (-4,0);
				\draw [fill=Green!60] (2,0) -- (3,2) -- (4,2) -- (3,0) -- (2,0);
				
				\node at  (0,1.2) [fill=Blue!40] {$\scriptstyle Virtual\;Core\;1$};
				\node at  (3.1,1.2)  {$\scriptstyle VC2$};
				
				\node at  (-2.5,-0.8) {$\scriptstyle Physical\;Core\;1$};
				\node at  (1.5,-0.8) {$\scriptstyle Physical\;Core\;2$};
				
				\draw [rounded corners, color=Blue] (-4.3,-1.2) rectangle (4.3,4.4);
				\node at (-3,4) [color=Blue] {$\scriptstyle Virtual\;Cores$};
				
				\node at (-2.6,0.8) [color=white] {$\scriptstyle Threadlets$};
				\draw [->,line width=1pt,style=dashed,color=white] (-0.2,0.8) .. controls (-2,0.6) .. (-3.5,-0.2);
				\draw [->,line width=1pt,style=dashed,color=white] (0,0.8) .. controls (-1,0.6) .. (-1.5,-0.2);
				\draw [->,line width=1pt,style=dashed,color=white] (0.3,0.8) .. controls (0.6,0.6) .. (0.6,-0.2);
				\draw [->,line width=1pt,style=dashed,color=white] (0.4,0.8) .. controls (0.6,0.7) .. (1.2,-0.2);
				
				\draw [->,line width=1pt,style=dashed,color=white] (2.9,0.8) .. controls (2.7,0.7) .. (2.5,-0.2);
				
				
				\end{tikzpicture}
				\end{figure}

	\end{frame}
	
	

%Weglassen??
%\subsubsection{Threadlets}	
%\begin{frame}{Threadlets}
	%Singlethread wird durch den virtuellen Kern in mehrere Threadlets zerlegt und auf die einzelnen physischen Kerne verteilt
%\begin{itemize}
%	\item Flaschenhals Zuweisung von Befehlen an die Kerne
%	\item Durch Threadlets maximal 1 Taktzyklus
%	\item Datenaustausch zwischen Registern sehr komplex
%	\item Gute Überwachung und Koordination, da sonst Datenverlust
%\end{itemize}

%\end{frame}	
	
%		\subsection{Pipeline}
%\begin{frame}
%\frametitle{Pipeline}
%	\begin{center}
%	Pipeline eventuell einbauen und/oder weglassen? guter übergang eventuell zu benchmark.
%	\end{center}
%\end{frame}

	
	\section{Benchmarks}
	\begin{frame}
		\frametitle{Benchmarks}
		%\begin{center}

%\begin{figure}%
%\begin{tikzpicture}
%	\node at (0,0) {\includegraphics[width=1\textwidth]{bilder/bench_1}};
%	\draw [color=Red, line width=1pt] (-0.5,-1.52) rectangle (0.7,1.6);
%\end{tikzpicture}
%\end{figure}
	%\end{center}
	\begin{center}
		
	\footnotesize
	\begin{tabular}{m{1cm}|>{\columncolor{lightgray!40}}m{0.8cm}|m{0.8cm}|>{\columncolor{lightgray!40}}m{0.8cm}|m{0.8cm}|>{\columncolor{lightgray!40}}m{0.9cm}|m{0.8cm}}
	\cline{6-6} & \textbf{Intel Atom 1Core} & \textbf{ARM A15 1Core} & \textbf{ARM A57 1Core} & \textbf{Apple A7 1Core} &\textbf{Soft Machine 2VCore Proto}  \tikzmark{left} & \textbf{Intel Haswell 1Core}\\
	\cline{6-6}
	\hline
	Compiled Code & 32-Bit & 32-Bit & 32 Bit & 32-Bit & 32-Bit & 64-Bit\\
	\hline
	Cache & 2M & 1M & 2M & 1M + 4M & 1M & 2M\\
	\hline
	Out-of-Order & 2-Way & 3-Way & 5-Way & 6-Way & 4-Way & 8-Way\\
	\hline
	%{\cellcolor[gray]{.8}}
	\rowcolor{Blue!20} {\cellcolor{white}}IPC (SPEC06*) & \textbf{0.69} & \textbf{0.71} & \textbf{0.87} & \textbf{1.0} & \textbf{\small2.1} & \tikzmark{right} \textbf{1.39}\\
	\cline{6-6}
	\end{tabular}
	\DrawBox[thick]
	\end{center}

	
	\normalsize
	
	\vspace{1cm} *Spec2006 Test mit GCC 4.6 Compiler, normalisiert mit 32-Bit und 28nm (2014)
	\end{frame}
	
\begin{frame}
\frametitle{Benchmarks}
	\begin{center}
	\begin{figure}
	\footnote{http://www.softmachines.com/wp-content/uploads/2014/12/Soft-Machines-VISCtm-Architecture-Tech-Briefing-vF.pdf}
\includegraphics[width=0.95\textwidth]{bilder/bench}
\end{figure}
	\end{center}
\end{frame}

\begin{frame}
\frametitle{Benchmarks}
	\begin{center}
	\begin{figure}
	\footnote{http://www.softmachines.com/wp-content/uploads/2015/10/Soft-Machines-VISC-Linley-PC-7Oct-2015-FINAL.pdf}
\includegraphics[width=0.95\textwidth]{bilder/bench_3}
\end{figure}
	\end{center}
\end{frame}

%\begin{frame}
	%\frametitle{Benchmarks}
	%\begin{itemize}
		%\item VISC-Zweikern Prozessor
		%\item unterschiedliche Prozessoren zum Vergleichen normalisiert
		%\item SPEC2006 mit 32-Bit, 16nm Kerne und GCC 4.6 Compiler
		%\item 1.7-fache Leistungssteigerung bei gleichem Energieverbrauch
	%	\item Serverleistung mit Stromverbrauch im Mobile Sektor
	%\end{itemize}
%\end{frame}
	
\section{Vermarktung}	
\begin{frame}
	\frametitle{Vermarktung}
	Keine eigene Produktion von Soft Machines
	\begin{itemize}
		\item \textbf{Lizensierung} - Architektur, Implementierung oder Technologie
		\item \textbf{Individuallösungen} - Integration in bestehende Systeme 
	\end{itemize}
	
\end{frame}

\section{VISC Roadmap}
	\begin{frame}
		\frametitle{VISC Roadmap}
	\begin{figure}\footnote{http://www.softmachines.com/wp-content/uploads/2015/10/Soft-Machines-VISC-Linley-PC-7Oct-2015-Final-Presentation.pdf}
\includegraphics[width=0.95\textwidth]{bilder/roadmap}
\end{figure}
	\end{frame}
	



\section{Fazit}
\begin{frame}
\frametitle{Fazit}
	\begin{itemize}
	  \pro Singlethread Optimierung
		\pro 2-4x mehr Leistung/Watt durch bessere IPC-Werte
		\pro Kompatibel mit allen ISA's
		\pro Vielversprechende Investoren
		\pro Belebt den Prozessormarkt
		\con Benchmarks nur unter optimalen Bedinungen
		\con Bisher noch kein Lizenznehmer
		\con Releasetermin für Shasta schon überschritten
	\end{itemize}
\end{frame}

\begin{frame}
\frametitle{Fazit}
	\begin{itemize}
	  \item Falls alle Ankündigungen umgesetzt werden, wird VISC der Überflieger, bis dahin bleibt es allerdings ein theoretisches Gebilde
	\end{itemize}
\end{frame}


\end{document}

